\documentclass[11pt]{article}

\usepackage[margin=1in]{geometry}

\usepackage{mathpazo}

\newcommand{\gcc}{\mathrm{g~cm^{-3} }}
\newcommand{\castro}{{\sffamily{Castro}}}

\begin{document}

\begin{center}
  {\Large Notes on the Radiation-Driven Planet Convection Problem}
\end{center}


\section{\castro\ Solver Details}


\section{Opacity}

YM10 suggest the form of the opacity as (Eq.\ 3):
\begin{equation}
  \tilde{\kappa} = \tilde{\kappa}_0 \left (\frac{P}{P_\mathrm{kb}} \right)
\end{equation}
with $\tilde{\kappa} = 0.18~\mathrm{cm^2~g^{-1}}$, and $P_\mathrm{kb}
= 1000~\mathrm{bar} = 10^9~\mathrm{erg~cm^{-3}}$.  Note that this form
of opacity has different units than what \castro\ uses.  For \castro\,
opacity has units of $\mathrm{cm^{-1}}$, so we have:
\begin{equation}
  \kappa = \rho \tilde{\kappa} = \tilde{\kappa}_0 \rho \left ( \frac{P}{P_\mathrm{kb}} \right )
\end{equation}

\castro\ takes the opacity as a simple power law of density and temperature (without any
reference values for scaling).  We replace the pressure using the EOS:
\begin{equation}
  P = \frac{\rho k T}{\mu m_p}
\end{equation}
where $\mu = 2.34$ as specified in YM10, and $T$ is understood to be
the gas temperature.  Substituting this in, we have:
\begin{equation}
  \kappa = \tilde{\kappa}_0 \rho^2 \frac{k T}{\mu m_p P_\mathrm{kb}}
\end{equation}

Since \castro\ wants $\kappa = \kappa_0 \rho^m T^{-n}$, we evaluate
this with $\rho = 1~\gcc$ and $T = 1~\mathrm{K}$:
\begin{equation}
  \kappa = 6.35 \times 10^{-3} \, \rho^2 T~\mathrm{cm^{-1}}
\end{equation}
This can be set in \castro\ as:
\begin{verbatim}
radiation.const_kappa_p = 6.35e-3
radiation.kappa_p.exp_m = 2
radiation.kappa_p_exp_n = -1
\end{verbatim}




\end{document}
